\documentclass[10pt,a4paper]{article}
\usepackage[utf8]{inputenc}
\usepackage[T1]{fontenc}
\usepackage[italian]{babel}
\usepackage{amsmath}
\usepackage{amsfonts} 
\usepackage{amssymb}
\usepackage{amsthm}
\usepackage{physics}
\usepackage{tikz}
\usetikzlibrary{backgrounds}
\usetikzlibrary{calc}
\tikzset{>=latex} % for LaTeX arrow head
\usepackage{xcolor}
\colorlet{veccol}{green!45!black}
\colorlet{myred}{red!90!black}
\colorlet{myblue}{blue!90!black}
\colorlet{mypurple}{blue!50!red!80!black!80}
\tikzstyle{vector}=[->,very thick,veccol]
\usepackage{pgfplots} % for the axis environment
\usetikzlibrary{calc} % to do arithmetic with coordinates
\usetikzlibrary{angles,quotes} % for pic
\usetikzlibrary{arrows.meta} % for arrow size
\usetikzlibrary{bending} % for arrow head angle
\tikzstyle{bend>}=[-{Latex[flex'=1,length=3,width=2.5]}]
\tikzstyle{bend<}=[{Latex[flex'=1,length=3,width=2.5]}-]
\usetikzlibrary{arrows.meta}
\tikzstyle{thin arrow}=[dashed,thin,-{Latex[length=4,width=3]}]
\usepackage[left=2cm,right=2cm,top=2cm,bottom=2cm]{geometry}
\author{Elisa Caruso}



\newtheorem{definition}{Definizione}[section]

\newtheorem{theorem}{Teorema}[section]
\newtheorem{lemma}[theorem]{Lemma}

\newcommand\normh[1]{\left\lVert#1\right\rVert}

\begin{document} 

\section{Introduzione}

Devo dire che è una dim geometrica\\
Devo dire che uso la dim di Buser e faccio solo i primi due teoremi di Bieb \\
Devo enunciare i teoremi di Bieb \\



\section{Considerazioni preliminari}
In questa sezione vengono enunciati e dimostrati gli strumenti fondamentali necessari per dare una dimostrazione geometria dei primi due teoremi di Bieberbach. \\
Considero lo spazio vettoriale n-dimensionale $\mathbb{R}^{n}$  con il prodotto scalare definito $\forall
 \vb{x} , \vb{y} \in \mathbb{R}^{3}$ da 

\[
  \vb{x} \cdot \vb{y} =  \sum_{i=1}^{n} x_i y_i 
\]
e la norma associata 
\[
  \norm{\vb{x}} =  \sqrt{\vb{x} \cdot \vb{x}}
\]
Definisco la distanza fra due vettori e l'angolo fra essi. \\
$ \forall \vb{x}, \vb{y} \in \mathbb{R}^n $  
\[ d(\vb{x}, \vb{y} ) = \norm{\vb{x-y}}\] 
\[ \angle (\vb{x},\vb{y}) = arccos \big( \frac{\vb{x} \cdot \vb{y}}{\norm{\vb{x}} \norm{\vb{y}}} \big) \in [0, \pi] \]
Imponendo una metrica sullo spazio vettoriale $\mathbb{R}^n$ ottengo lo spazio euclideo $\mathbb{E}^n$.

\begin{definition}
	Un'isometria di $\mathbb{E}^n$  è un funzione $ \alpha : \mathbb{R}^{n} \longrightarrow \mathbb{R}^{n} $  tale che $\forall \vb{x,y} \in \mathbb{R}^n $ vale 
	\[ d(\alpha(\vb{x}), \alpha(\vb{y})) = d(\vb{x} , \vb{y} )\]
\end{definition}
Il seguente teorema è enunciato senza dimostraione in quanto si tratta di un risultato classico dell'algebra lineare. 

\begin{theorem}
Data un'isometria di $\mathbb{E}^n $, questa può essere scritta in modo unico come composizione di una rotazione e di una traslazione
\[ \alpha : \mathbb{R}^{n} \longrightarrow \mathbb{R}^{n} \]
\[\vb{x} \longmapsto \vb{Ax + a} \]
dove $\vb{A} = rot( \alpha ) \in O(n) $ è la componente di rotazione di $\alpha$ \\
e $\vb{a} = trans( \alpha ) \in \mathbb{R}^{n} $ è la componente di traslazione dell'isometria. 
\end{theorem}

\begin{lemma}
L'insieme delle isometrie di  $\mathbb{E}^{n} $  è un gruppo rispetto all'operazione di composizione di applicazioni lineari. Lo chiamiamo $Isom(n)$.
\end{lemma}

\begin{definition}
	Comunque prese  $ \alpha , \beta \in Isom(n)$ posso scriverle come $\alpha : \vb{x} \longmapsto \vb{Ax + a}$ e $\beta : \vb{x} \longmapsto \vb{Bx + b}$ \\
	Definisco il loro commutatore come  $ [ \alpha , \beta] = \alpha \beta \alpha^{-1} \beta^{-1}$. 
\end{definition}

\begin{lemma}
Comunque prese  $ \alpha , \beta \in Isom(n)$ valgono le due seguenti uguaglianze:
\begin{equation}
rot([ \alpha	, \beta ]) = \vb{[A, B]} = \vb{ABA^{-1}B^{-1}}
\end{equation}  
\begin{equation}
trans([ \alpha	, \beta ]) = \vb{(A-id)b+(id-[A,B])b+A(id-B)A^{-1}a }
\end{equation} 
\end{lemma}

\begin{proof}
	$ [ \alpha, \beta ](\vb{x}) = ( \alpha \beta \alpha^{-1} \beta^{-1} ) (\vb{x})$. \\
	Dato che $ \alpha : \vb{x} \longmapsto \vb{Ax + a}$ , allora sicuramente  $\alpha^{-1}: \vb{y} \longmapsto \vb{A^{-1}(y-a)}$.  \\
	Allo stesso modo, dato che $ \beta : \vb{x} \longmapsto \vb{Bx + b}$ , so che $\beta^{-1}: \vb{y} \longmapsto \vb{B^{-1}(y-b)}$. 

\begin{equation*}
\begin{split}
[ \alpha, \beta ](\vb{x}) = (\alpha \beta \alpha^{-1} \beta^{-1} ) (\vb{x})
& = (\alpha \beta \alpha^{-1})( \vb{B^{-1}(x-b)}) = \\ 
& = (\alpha \beta \alpha^{-1})( \vb{B^{-1}x- B^{-1}b }) = \\ 
& = (\alpha \beta)(\vb{A^{-1}( B^{-1}x- B^{-1}b -a)})= \\
& =  (\alpha \beta)(\vb{A^{-1} B^{-1}x- A^{-1}B^{-1}b -A^{-1}a})  = \\
& = (\alpha)(\vb{B(A^{-1} B^{-1}x- A^{-1}B^{-1}b -A^{-1}a) +b}) = \\
& = (\alpha)(\vb{BA^{-1} B^{-1}x-BA^{-1}B^{-1}Su questo spazio db -BA^{-1}a +b}) = \\
& = \vb{A(BA^{-1} B^{-1}x-BA^{-1}B^{-1}b -BA^{-1}a +b) + a} = \\
& = \vb{A(BA^{-1} B^{-1}x-BA^{-1}B^{-1}b -BA^{-1}a +b) + a} = \\
& = \vb{ABA^{-1} B^{-1}x-ABA^{-1}B^{-1}b -ABA^{-1}a +Ab + a }
\end{split}
\end{equation*}

Quindi ho che $ rot([ \alpha , \beta ]) = \vb{ABA^{-1} B^{-1} }$ e
\begin{equation*}
\begin{split}
trans([ \alpha, \beta ]) & = \vb{-ABA^{-1}B^{-1}b -ABA^{-1}a +Ab + a }= \\
& = \vb{Ab + a -ABA^{-1}B^{-1}b -ABA^{-1}a }= \\
& = \vb{(A-id)b + b + a -ABA^{-1}B^{-1}b -ABA^{-1}a} = \\
& = \vb{(A-id)b + (id-[A,B])b + a - ABA^{-1}a} = \\
& = \vb{(A-id)b + (id-[A,B])b + (id- ABA^{-1})a }= \\
& = \vb{(A-id)b + (id-[A,B])b + A(id- B)A^{-1}a} 
\end{split}
\end{equation*}
	
\end{proof}

\begin{definition}
Comunque preso  $  \vb{A} \in O(n) $  definisco
\[ m(\vb{A}) = \max \bigg\{ \frac{\norm{\vb{Ax-x}}}{\norm{\vb{x}}} \bigg|  \vb{x} \in \mathbb{R}^{n}-\vb{0} \bigg\} \]
\end{definition}

\begin{lemma}
La precedente è una buona definizione e inoltre vale la seguente uguaglianza: 
\[ m(\vb{A}) = \max \bigg\{ \norm{\vb{Ax-x}} \bigg|  \vb{x} \in \mathbb{R}^{n} \land \norm{\vb{x}} = 1 \bigg\} \]
\end{lemma}

\begin{proof}
$ \forall \vb{x} \in \mathbb{R}^{n}-\vb{0}$  $ \exists \vb{y} \in \mathbb{R}^{n}-\vb{0} $ tale che \\$\vb{x} = \lambda \vb{y} , \lambda \in \mathbb{R} \land \norm{\vb{y}} = 1 $. 

\[ \frac{\norm{\vb{Ax-x}}}{\norm{\vb{x}}} = \frac{\norm{\vb{A}\lambda \vb{y}-\lambda \vb{y}}}{\norm{\lambda \vb{y}}} =   \frac{|\lambda| \norm{\vb{Ay-y}}}{|\lambda| \norm{\vb{ y}}} = \norm{\vb{Ay-y}} \]

\[ m(\vb{A}) = \max \bigg\{ \norm{\vb{Ax-x}} \bigg|  \vb{x} \in \mathbb{R}^{n} \land \norm{\vb{x}} = 1 \bigg\}\] L'insime $\vb{x} \in \mathbb{R}^{n} \land \norm{\vb{x}} = 1 $  è un compatto in  $\mathbb{R}^{n}$, quindi per il teorema di Weierstrass esiste una $\vb{x} \in \mathbb{R}^{n} \land \norm{\vb{x}} = 1$ che mi verifica il max. 
\end{proof}

\begin{lemma}
Comunque preso $ \vb{x} \in \mathbb{R}^{n}$  vale  \[ \norm{\vb{Ax-x}} \leq m(\vb{A}) \norm{\vb{x}} \]
\end{lemma}

\begin{proof}
Infatti vale $\forall \vb{x} \in \mathbb{R}^{n}-\vb{0}$  e vale anche per  $\vb{x}=0$
\end{proof}

\begin{definition}
\[ E_{\vb{A}} := \{ \vb{x} \in \mathbb{R}^{n} :  \norm{\vb{Ax - x}} = m(\vb{A}) \norm{\vb{x}}\} \]
\end{definition}

\begin{lemma}
$E_A$ è un sottospazio di $\mathbb{R}^{n}$ non banale ed A-invariante
\end{lemma} 

\begin{proof}
\begin{itemize}
\item $\vb{0} \in E_{\vb{A}} $  
\item $\forall \vb{x} \in E_{\vb{A}}, \forall \lambda \in \mathbb{R}$ \\
$\vb{x} \in E_{\vb{A}} \Longrightarrow \norm{\vb{Ax-x}} = m(\vb{A})\norm{\vb{x}} \Longrightarrow \norm{\vb{A} \lambda \vb{x} -\lambda \vb{x}}= | \lambda | m(\vb{A}) \norm{\vb{x}}  \Longrightarrow \lambda \vb{x} \in E_{\vb{A}} $
\item $\forall \vb{x, y} \in E_{\vb{A}} $ voglio verificare che $\vb{x+y, x-y} \in E_{\vb{A}}$ \\
$\vb{x} \in E_{\vb{A}} \Longrightarrow \norm{\vb{Ax-x}} = m(\vb{A})\norm{\vb{x}}$ \\
$\vb{y} \in E_{\vb{A}} \Longrightarrow \norm{\vb{Ay-y}} = m(\vb{A})\norm{\vb{y}}$ 

\begin{equation} 
\begin{split}
& \norm{\vb{A(x+y) - (x+y)}}^2 + \norm{\vb{A(x-y) - (x-y)}}^2  = \norm{\vb{Ax - x + Ay -y}}^2 + \norm{\vb{Ax - x -(Ay -y)}}^2 = \\
& = 2 \norm{\vb{Ax-x}}^2 + 2 \norm{\vb{Ay -y}}^2  = 2( \norm{\vb{Ax-x}}^2 + \norm{\vb{Ay-y}}^2) = 2m(\vb{A})^2(\norm{\vb{x}}^2 + \norm{\vb{y}}^2) 
\end{split}
\end{equation}
Da (3) segue 
\begin{equation} 
\begin{split}
2m(\vb{A})^2(\norm{\vb{x}}^2 + \norm{\vb{y}}^2) & = \norm{\vb{A(x+y) - (x+y)}}^2 + \norm{\vb{A(x-y) - (x-y)}}^2 \leq m(\vb{A})^2(\norm{\vb{x+y}}^2 + \norm{\vb{x-y}}^2) \\ 
& = m(\vb{A})^2(\norm{\vb{x+y}}^2 + \norm{\vb{x-y}}^2)  = 2m(\vb{A})^2(\norm{\vb{x}}^2 + \norm{\vb{y}}^2)
\end{split}
\end{equation}
Quindi il segno di disuguaglianza in (4) è un'uguaglianza, in particolare 
\[   \norm{\vb{A(x+y) - (x+y)}}^2 + \norm{\vb{A(x-y) - (x-y)}}^2  = m(\vb{A})^2(\norm{\vb{x+y}}^2 + \norm{\vb{x-y}}^2)  \]
Spostando i termini da parte a parte ottengo
\[ \norm{\vb{A(x+y) - (x+y)}}^2 - m(\vb{A})^2 \norm{\vb{x+y}}^2=  m(\vb{A})^2 \norm{\vb{x-y}}^2 - \norm{\vb{A(x-y) - (x-y)}}^2 \]
\begin{equation} 
\begin{split}
\bigg( \norm{\vb{A(x+y) - (x+y)}} & + m(\vb{A})\norm{\vb{x+y}} \bigg) \bigg( \norm{\vb{A(x+y) - (x+y)}} - m(\vb{A}) \norm{\vb{x+y}} \bigg) = \\
& -\bigg( \norm{\vb{A(x-y) - (x-y)}} + m(\vb{A})\norm{\vb{x-y}} \bigg) \bigg( \norm{\vb{A(x-y) - (x-y)}} - m(\vb{A}) \norm{\vb{x-y}} \bigg)
\end{split}
\end{equation}
Distinguiamo alcuni casi: 
\begin{itemize}
\item Se $\vb{A} = \vb{id} \Rightarrow m(\vb{A}) = 0 \Rightarrow E_{\vb{A}} = \mathbb{R}^{n}$  ed in quel caso ho chiuso la dimostrazione.
\item Se $\vb{x} = \vb{y}$ o $\vb{x} = \vb{-y}$  ho già chiuso la dimostrazione per il punto precedente (rispettivamente con $\lambda = +1$ e $\lambda = -1$). 
\item Nei restanti casi posso osservare che nell'equazione (5) il primo fattore a sinistra dell'uguaglianza è è strettamente positivo, mentre il secondo fattore è $\leq 0 $. Allo stesso modo a destra dell'uguaglianza ho un meno che mi modifica il segno del prodotto; il primo fattore è strettamente positivo ed il secondo è $\leq 0 $. \\ L'uguaglianza in (5) deve quindi per forza coincidere con $0=0$ e questo mi implica 
\begin{equation*}
  \left\{
    \begin{aligned}
      & \norm{\vb{A(x+y) - (x+y)}} = m(\vb{A})\norm{\vb{x+y}} \\
      & \norm{\vb{A(x-y) - (x-y)}} = m(\vb{A})\norm{\vb{x-y}} 
    \end{aligned}
  \right.
\end{equation*}

$\Longrightarrow \vb{x+y, x-y} \in E_{\vb{A}} $ 
\end{itemize}  
Ho concluso la dimostrazione del fatto che $E_{\vb{A}}$ è un sottospazio vettoriale di $\mathbb{R}^n $

\item Mostro che $E_{\vb{A}}$ è non banale. 
\[ m(\vb{A}) = \max \bigg\{ \norm{\vb{Ax-x}} \bigg|  \vb{x} \in \mathbb{R}^{n} \land \norm{\vb{x}} = 1 \bigg\} \] 
 L'insieme degli $\vb{x} \in \mathbb{R}^{n} \land \norm{\vb{x}} = 1$ è un compatto; l'applicazione $ \vb{x} \mapsto \norm{\vb{Ax-x}}$ è continua in quanto composizione di funzioni continue, quindi sicuramente esiste un $\vb{x} \in \mathbb{R}^{n} \land \norm{\vb{x}} = 1$  che mi verifica il massimo. In particolare $\vb{x} \neq \vb{0}$  (perché ha norma 1) e 

\[ \norm{\vb{Ax - x}} = m(\vb{A}) = m(\vb{A})\norm{\vb{x}} \Longrightarrow \vb{x} \in E_{\vb{A}}\]


\item Mostro che $E_{\vb{A}}$ è $\vb{A}$-invariante. \\
$\forall \vb{x} \in E_{\vb{A}}$  voglio mostrare che $\vb{Ax} \in E_{\vb{A}}$ \\
$\vb{x} \in E_{\vb{A}} \Longrightarrow \norm{\vb{Ax-x}} = m(\vb{A})\norm{\vb{x}} $ \\
Dato che $\vb{A} \in O(n)$ ho la seguente catena di uguaglianze:\\
$ \norm{\vb{A(Ax) - Ax }} =  \norm{\vb{A(Ax-x)}} = \norm{\vb{Ax - x}} = m(\vb{A})\norm{\vb{x}} = m(\vb{A})\norm{\vb{Ax}} \Longrightarrow \vb{Ax} \in E_{\vb{A}}$
\end{itemize} 
\end{proof}



\begin{lemma}
$E_{\vb{A}}$ sottospazio vettoriale di $\mathbb{R}^n $ non banale, quindi posso definire il suo complemento ortogonale $E^{\perp}_{\vb{A}} \neq \mathbb{R}^n$ ed anche questo è $\vb{A}$-invariante. 
\end{lemma} 

\begin{definition}

\[ m^{\perp}(\vb{A}) = \begin{cases} 
      \max \bigg\{ \frac{\norm{\vb{Ax-x}}}{\norm{\vb{x}}} \bigg|  \vb{x} \in E^{\perp}_{\vb{A}} - \vb{0} \bigg\}  &  se   E^{\perp}_{\vb{A}} \neq {\vb{0}} \\
      0 & se   E^{\perp}_{\vb{A}} = {\vb{0}} \\
      
   \end{cases}
\]

\end{definition}

\begin{lemma}
$m^{\perp}(\vb{A}) < m(\vb{A})$ se $\vb{A \neq id} $\\
$m^{\perp}(\vb{A}) = m(\vb{A}) = 0$ se $\vb{A = id}$

\end{lemma}

\begin{proof}
\begin{itemize}
\item Se $\vb{A = id} \Longrightarrow m(\vb{A}) = 0 \land E_{\vb{A}} = \mathbb{R}^n$. Quindi $E^{\perp}_{\vb{A}} = {0} \Longrightarrow m^{\perp}(\vb{A}) = 0$

\item Se $\vb{A \neq id}$ \\
\[ m(\vb{A}) = \max \bigg\{ \frac{\norm{\vb{Ax-x}}}{\norm{\vb{x}}} \bigg|  \vb{x} \in \mathbb{R}^{n}-\vb{0} \bigg\} 
\geq \max \bigg\{ \frac{\norm{\vb{Ax-x}}}{\norm{\vb{x}}} \bigg|  \vb{x} \in E^{\perp}_{\vb{A}} - \vb{0} \bigg\} =  m^{\perp}(\vb{A}) \]
Se valesse $m(\vb{A}) = m^{\perp}(\vb{A}) \Longrightarrow \exists \vb{x} \in E^{\perp}_{\vb{A}} - \vb{0} : \norm{\vb{Ax-x}} = m(\vb{A})\norm{\vb{x}} \Longrightarrow \vb{x} \in E_{\vb{A}}$  ma questo implica $\vb{x} \in E_{\vb{A}} \cap E^{\perp}_{\vb{E}} = {\vb{0}} \Longrightarrow \vb{x = 0} $  ma questo è assurdo perché ho imposto che $x\vb{ \neq 0}$ 
\end{itemize}
\end{proof}

\begin{lemma}
$\forall \vb{x} \in \mathbb{R}^n $ posso scrivere $\vb{x}$ in modo unico in decomposizione ortogonale. \\
$\vb{x} = \vb{x}^E + \vb{x}^{\perp}$ dove $\vb{x}^E \in E_{\vb{A}}$ e $\vb{x}^{\perp} \in E^{\perp}_{\vb{A}}$ \\
Valgono inoltre le proprietà:
\[ \norm{\vb{Ax}^E - \vb{x} }= m(\vb{A})\norm{\vb{x}^E} \] \[\norm{\vb{Ax}^{\perp} - \vb{x}} \leq m(\vb{A}) \norm{\vb{x}^{\perp}} \]
\end{lemma}

\begin{lemma}
 $\forall \vb{ A, B} \in O(n)$ vale la disuguaglianza $ m([\vb{A,B}]) \leq 2m(\vb{A})m(\vb{B})$ 
\end{lemma}

\begin{proof}



\[ [\vb{A,B}] - \vb{id} = \vb{ABA^{-1}B^{-1} - id }= \vb{(AB -BA)A^{-1}B^{-1} }= [ \vb{ (A-id)(B-id) - (B-id)(A-id)}] \vb{A^{-1}B^{-1}} = \]
 \[  = \vb{( A-id)(B-id) A^{-1}B^{-1} -  (B-id)(A-id) A^{-1}B^{-1}} \]
 
 
 \begin{equation} 
\begin{split}
\norm{\vb{([A,B] - id )x}} & \leq \norm{\vb{ ( A-id)(B-id) A^{-1}B^{-1}x}} +  \norm{\vb{(B-id)(A-id) A^{-1}B^{-1}x}} \leq \\
& \leq m(\vb{A}) \norm{\vb{(B-id) A^{-1}B^{-1}x}} + m(\vb{B}) \norm{\vb{(A-id) A^{-1}B^{-1}x}} \leq \\ 
& \leq m(\vb{A})m(\vb{B})\norm{\vb{A^{-1}B^{-1}x}} + m(\vb{B})m(\vb{A}) \norm{\vb{A^{-1}B^{-1}x}}
\end{split}
\end{equation}

Dato che $ \vb{A,B} \in O(n)$ posso scrivere $\norm{\vb{A^{-1}B^{-1}x}} = \norm{\vb{x}} $ , quindi 

\begin{equation}
  \norm{\vb{([A,B] - id )x}}  \leq 2m(\vb{A})m(\vb{B})\norm{\vb{x}}
\end{equation} 
Quindi segue immediatametne la tesi. 
\end{proof}
Si richiede, di solito, quando si parla di gruppi cristallografici, che abbiano un dominio fondamentale discreto e compatto. In particolare in questo caso utilizziamo la seguente definizione:

\begin{definition}
Un gruppo $\Gamma$ di isometrie in $\mathbb{R}^{n}$ è detto cristallografico se valgono le seguenti condizioni:
\begin{itemize}
	\item $ \forall t \in \mathbb{R} : t > 0 $  esistono solo un numero finito di $\alpha \in \Gamma $  tali che  $|a| \leq t$

	\item $ \exists d \in \mathbb{R}: \forall \vb{x} \in \mathbb{R}^n   \exists \alpha \in \Gamma : \norm{\vb{a-x}}\leq d $
\end{itemize}
\end{definition}

Sia ora $\Gamma$ un gruppo cristallografico in $\mathbb{R}^n$ 

\begin{theorem}{Mini Bieberbach}
 $ \forall \vb{u} \in \mathbb{R}^n : \norm{\vb{u}} = 1,   \forall \epsilon , \delta >0 $  $   \exists \beta \in \Gamma $ che soddisfa  
\begin{itemize}
\item $\vb{b} \neq 0$ 
\item $ \angle (\vb{u, b}) \leq \delta $
\item $ m(\vb{B}) \leq \epsilon$ 
\end{itemize}
\end{theorem}

\begin{proof}
\begin{itemize}

\item  Per la seconda proprietà dei gruppi cristallografici so che $ \exists d \in \mathbb{R} : \forall k \in \mathbb{N} $ $  \exists \beta_k \in \Gamma : $ \\
\[ \beta_k \vb{x} = \vb{B_k x + b_k}  \]
\[ \norm{\vb{b_k - ku }} \leq d \] 
Se $k \longrightarrow \infty$ allora sicuramente $ \norm{\vb{b_k}} \longrightarrow \infty$.  \\
Infatti se, per assurdo, questo non fosse vero $\exists M \in \mathbb{R} : \norm{\vb{b_k}} \leq M $   $  \forall k \in \mathbb{N} $

\[ d \geq \norm{\vb{b_k - ku}} \geq | \norm{\vb{b_k}} - k | \]
Nella disequazione  precedente, se la successione è limitata allora l'ultimo termine diverge ma questo è assurdo perché è maggiorato da $d$. 

\item Ho definito la successione ${ \{\beta_k \}}_{k \in \mathbb{N}}$  con  $\beta_k \vb{x = B}_k \vb{x + b}_k$   $\beta_k \in \Gamma \Longrightarrow \vb{B}_k \in O(n)$  \\
$O(n)$  è compatto e ${ \{\vb{B}_k \} }_{k \in \mathbb{N}}$ successione in  $O(n)$  ammette quindi almeno un punto di accumulazione per il teorema di Bolzano-Weierstrass. \\
Estraggo da ${ \{\vb{B}_k \}}_{k \in \mathbb{N}}$  una sottosuccessione  $ { \{ \vb{A}_k \} }_{k \in \mathbb{N}} = { \{ \vb{B}_k_j \} }_{j \in \mathbb{N}}$  convergente.  \\
La funzione  $ m: O(n) \longrightarrow \mathbb{R}$  è continua in quanto composizione di funzioni continue ($ \vb{A} \mapsto \max{\norm{\vb{Ax-x}}} $), quindi con $i,j \in \mathbb{B} \longrightarrow \infty $ sicuramente $m(\vb{C}_j \vb{C}^{-1}_i) \longrightarrow m(\vb{id}) = 0 $


\item Considero ora
\[ \angle (\vb{u , b_k}) = \angle (\vb{ku, b_k}) = : \theta_k \]


\[ \frac{\norm{\vb{b_k}}}{sin(\gamma_k)} = \frac{\norm{\vb{b_k} - k\vb{u}}}{sin(\theta_k)}  \leq \frac{d}{sin(\theta_k)} \]

\[  sin(\theta_k) \leq \frac{sin(\gamma_k)d}{\norm{\vb{b_k}}} \longrightarrow 0 \Longrightarrow \theta_k \longrightarrow 0 \]

\begin{tikzpicture}[line cap=round]
  \coordinate (O) at (0,0);
  \coordinate (A) at (6, 0);
  \coordinate (B) at (5,2.4);
  \coordinate (A-B) at ($(A)-(B)$);
  

  \draw[vector,myred] (O) -- (A) node[midway,below] {$k \vb{u}$};
  \draw[vector,myblue] (O) -- (B) node[midway,above left=-2] {$\vb{b_k}$};
  \draw[vector,mypurple] (A) -- (B) node[above right=-3] {$\vb{b_k}-k\vb{u}$};
  \pic[draw,"$\theta_k$",angle radius=20,angle eccentricity=1.25] {angle=A--O--B}; 
  \pic[draw,"$\gamma_k$",angle radius=20,angle eccentricity=1.25] {angle=B--A--O};  
\end{tikzpicture}
 
\[ \Longrightarrow \exists i, j \in \mathbb{N} : \angle(u, \vb{b}_j ) \leq \frac{\delta}{4} \]

\end{itemize}



\begin{tikzpicture}[line cap=round]
  
  \coordinate (O) at (0,0);
  \coordinate (A) at (12,0);
  \coordinate (B) at (3,2);
  \coordinate (C) at ($(O)-(B)$);
  \coordinate (U) at (2, -3 );
  \coordinate (X) at ($(A)+(C)$);
  

  \draw[vector] (O) -- (A) node[midway,below] {$\vb{b}_j$};
  \draw[vector] (O) -- (B) node[midway,above left = -2] {$\vb{B_j B_i^{-1} b_i}$};
  \draw (A) circle (4);
  \draw[vector] (O) -- (U) node[midway, below] {$ \vb{u} $};
  \draw[vector] (A) -- (X);
  \draw[vector] (O) -- (X);
\end{tikzpicture}


\begin{tikzpicture}
  \def\r{2} % radius
  \def\q{-7} % distance center-external point q = |OQ|
  \def\x{{\r^2/\q}} % Q x coordinate
  \def\y{{\r*sqrt(1-(\r/\q)^2}} % Q y coordinate
  \coordinate (O) at (0,0); % circle center O
  \coordinate (Q) at (\q,0); % external point Q
  \coordinate (P) at (\x,\y); % point of tangency, P
  \coordinate (X) at (1, 1);
  \draw[blue,thick] (O) circle(\r);
  \draw[green,thick] (Q) -- (P);
  \draw[green,thick] (P) -- (O);
  \draw[green,thick] (O) -- (Q);
  \draw (O) -- (X);
  \draw (Q) -- (X);
\end{tikzpicture}


\begin{tikzpicture}
  \coordinate (O) at (0,0);
  \coordinate (A) at (3,7);
  \coordinate (B) at (0,4);
  \coordinate (C) at (1.5,5);
  \draw[dashed] (-5,0) -- (5,0)node[right] {$E^{\perp}_{\vb{A}}$};
  \draw[dashed] (1.5,0) |- (0,5);
  \draw[vector, red, thick] (O) -- (1.5,0)node[below] {$\vb{b}^{\perp}$};
  \draw[vector, red, thick] (O) -- (0,5)node[left] {$\vb{b}^{E}$};
  \draw[thick] (0,-2) -- (0,7)node[above] {$E_{\vb{A}}$};
  \draw[vector] (O) -- (B)node[right] {$\vb{u}$};
  \draw[vector] (O) -- (C)node[right] {$\vb{b}$};
  \draw[blue, thick] (0,0) -- (3,7)node[right];
  \draw[blue, thick] (0,0) -- (-3,7)node[right];
  \filldraw[black] (0,0) circle (2pt);
  \pic[draw,"$\delta$",angle radius=30,angle eccentricity=1.25] {angle=A--O--B};  
  \begin{scope}[on background layer]
		\fill[blue!20] (0,0) -- (-3,7) -- (3,7) -- cycle;
  \end{scope}
\end{tikzpicture}



\end{proof}

\end{document}
