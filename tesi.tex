\documentclass[12pt,a4paper]{book}
\usepackage[utf8]{inputenc}
\usepackage[T1]{fontenc}
\usepackage[italian]{babel}
\usepackage{amsmath}
\usepackage{amsfonts} 
\usepackage{amssymb}
\usepackage{amsthm}
\usepackage{physics}
\usepackage{tikz}
\usepackage{ wasysym }
\usepackage{caption}
\usepackage[justification=centering]{caption}
\usetikzlibrary{backgrounds}
\usetikzlibrary{calc}
\tikzset{>=latex} % for LaTeX arrow head
\usepackage{xcolor}
\colorlet{veccol}{green!45!black}
\colorlet{myred}{red!90!black}
\colorlet{myblue}{blue!90!black}
\colorlet{mypurple}{blue!50!red!80!black!80}
\tikzstyle{vector}=[->,very thick,veccol]
\usepackage{pgfplots} % for the axis environment
\usetikzlibrary{calc} % to do arithmetic with coordinates
\usetikzlibrary{angles,quotes} % for pic
\usetikzlibrary{arrows.meta} % for arrow size
\usetikzlibrary{bending} % for arrow head angle
\tikzstyle{bend>}=[-{Latex[flex'=1,length=3,width=2.5]}]
\tikzstyle{bend<}=[{Latex[flex'=1,length=3,width=2.5]}-]
\usetikzlibrary{arrows.meta}
\tikzstyle{thin arrow}=[dashed,thin,-{Latex[length=4,width=3]}]
\usepackage[left=2cm,right=2cm,top=2cm,bottom=2cm]{geometry}
\author{Elisa Caruso}



\newtheorem{definition}{Definizione}[section]

\newtheorem{theorem}{Teorema}[section]
\newtheorem{lemma}[theorem]{Lemma}

\newcommand\normh[1]{\left\lVert#1\right\rVert}


\newcommand*\quot[2]{{^{\textstyle #1}\big/_{\textstyle #2}}}

\begin{document} 

\chapter*{Introduzione e motivazioni}

%Devo dire che è una dim geometrica\\
%Devo dire che uso la dim di Buser e faccio solo i primi due teoremi di Bieb \\
%Devo enunciare i teoremi di Bieb \\


\chapter{Risultati preliminari}
In questo capitolo enunciamo alcune definizioni e risultati fondamentali per approcciare la dimostrazione geometrica dei primi due teoremi di Bieberbach.  \\

Nella prima sezione riprendiamo la definizione di spazio euclideo $n$-dimensionale e del suo gruppo di isometrie, descriviamo una loro rappresentazione come composizione dell'applicazione di una matrice ortogonale e di una translazione; definiamo poi poi il coniugio di due isometrie. \\
La lunghezza e la direzzione del vettore translazione di una data isometria danno immediatamente informazioni su come questa agisce sui punti di $\mathbb{R}^n$; più complicato è invece capire, data una matrice ortogonale, come questa trasformi lo spazio. \\
Nella seconda sezione viene defnita quindi una funzione che stima per eccesso di quanto ogni matrice ortogonale si discosta dalla matrice identità; viene poi definita attraverso essa una scomposizione dello spazio in due spazi ortogonali; infine si dimostra un teorema che stima la "misura" del commutatore di due matrici a partire dalle loro "misure".  


\section{Spazio euclideo $\mathbb{E}^n$ ed isometrie}
Consideriamo lo spazio vettoriale n-dimensionale $\mathbb{R}^{n}$  con il prodotto scalare dato da 
\[
  \vb{x} \cdot \vb{y} =  \sum_{i=1}^{n} x_i y_i    \; \; \; \forall \vb{x} , \vb{y} \in \mathbb{R}^{n}
\]
e la norma ad esso associata              
\[
  \norm{\vb{x}} =  \sqrt{\vb{x} \cdot \vb{x}}     \; \; \; \; \; \forall \vb{x} , \vb{y} \in \mathbb{R}^{n}.
\]
Imponiamo sullo spazio la distanza associata al prodotto scalare 
\[ d(\vb{x}, \vb{y} ) = \norm{\vb{x-y}}           \; \; \; \forall \vb{x} , \vb{y} \in \mathbb{R}^{n}\] 
Imponendo questa metrica sullo spazio vettoriale $\mathbb{R}^n$ ottengo lo spazio euclideo $\mathbb{E}^n$. 
\begin{definition}
Dati $\vb{x, y} \in \mathbb{R}^n$ definisco l'angolo fra di essi $\angle (\vb{x},\vb{y})$ come 
\[ \angle (\vb{x},\vb{y}) = arccos \bigg( \frac{\vb{x} \cdot \vb{y}}{\norm{\vb{x}} \norm{\vb{y}}} \bigg) \in [0, \pi] \]
\end{definition} 
\'E noto che gli automorfismi di $\mathbb{R}^n$ (ovvero le applicazioni lineari biettive da $\mathbb{R}^n$ in sé stesso) formano un gruppo con l'operazione di composizione di applicazioni lineari isomorfo  a $ GL(n, \mathbb{R}) $, il gruppo delle matrici quadrate $n-$dimensionali con determinante non nullo. 
Se si considerano invece le mappe affini da $\mathbb{R}^n$ in sé stesso con il prodotto dato dalla composizione, si ottinene il gruppo affine $Aff(n, \mathbb{R})$; esso può essere rappresentato come il prodotto demidiretto 
\[  \mathbb{R}^{n} \rtimes GL(n, \mathbb{R})   \]
con l'operazione di composizione data da 
\[ (\vb{A, a}) \cdot (\vb{B, b}) = (\vb{ a + Ab, AB})  \; \; \forall \vb{A, B}  \in  GL(n, \mathbb{R}) \; \forall \vb{a, b} \in \mathbb{R}^n. \]
I teoremi di Bieberbach trattano in particolare i gruppi cristallograici, che sono dei particolari tipi di sottogruppi del gruppo delle isometrie.
\begin{definition}
	Un'isometria di $\mathbb{E}^n$  è un funzione $ \alpha : \mathbb{R}^{n} \longrightarrow \mathbb{R}^{n} $  tale che $\forall \vb{x,y} \in \mathbb{R}^n $ vale 
	\[ d(\alpha(\vb{x}), \alpha(\vb{y})) = d(\vb{x} , \vb{y} )\]
\end{definition} 
Il seguente teorema è enunciato senza dimostrazione in quanto si tratta di un risultato classico. 
\begin{theorem}
Data un'isometria di $\mathbb{E}^n $, questa può essere scritta in modo unico come composizione di una rotazione e di una traslazione rispetto alla base standard di $\mathbb{R}^n$
\[ \alpha : \mathbb{R}^{n} \longrightarrow \mathbb{R}^{n} \]
\[\vb{x} \longmapsto \vb{Ax + a} \]
dove $\vb{A} = rot( \alpha ) \in O(n) $ è detta componente di rotazione di $\alpha$ \\
e $\vb{a} = trans( \alpha ) \in \mathbb{R}^{n} $ è detta componente di traslazione di $\alpha$. 
\end{theorem}
In questa trattazione indicheremo spesso le isometrie secondo la rappresentazione fornita dal teorema precedente. In particolare scriveremo $\alpha = ( \vb{A, a})  $ per indicare che l'isometria $\alpha$ ha come componente di rotazione la matrice $\vb{A}$ e come componente di translazione il vettore $\vb{a}$. \\
Indichiamo l'insieme delle isometrie di  $\mathbb{E}^{n} $ come $Isom(n)$
\begin{lemma}
$Isom(n)$ è un gruppo con operazione di composizione di applicazioni; in particolare è un sottogruppo del gruppo affine e come tale può essere scritto come prodotto semidiretto nella forma
\[ Isom(n) \cong   O(n) \rtimes \mathbb{R}^{n} <  GL(n, \mathbb{R}) \rtimes\mathbb{R}^{n} 
\]
dove $O(n)$ indica il gruppo delle matrici ortogonali $n-$dimensionali, in particolare 
\[ O(n) = \{ \vb{A} \in GL(n, \mathbb{R}) \big| \vb{A A^{T}} = \vb{A^{T} A} = \vb{id} \}\]
\end{lemma}
\'E noto che le matrici ortogonali hanno la proprietà
\[ \norm{\vb{Ax}} = \norm{\vb{x}}   \; \; \; \; \forall x \in \mathbb{R}^n \; \forall A \in O(n)\]
All'interno di ogni gruppo è definito il commutatore di due elementi, posso quindi anche definirlo per $Isom(n)$.
\begin{lemma}
	Comunque prese  $ \alpha = (\vb{A,a}), \beta = (\vb{B,b}) \in Isom(n)$\\
	definisco il loro commutatore come  $ [ \alpha , \beta] = \alpha \beta \alpha^{-1} \beta^{-1}$. \\
Valgono le due seguenti uguaglianze:
\begin{equation}
rot([ \alpha	, \beta ]) = \vb{[A, B]} = \vb{ABA^{-1}B^{-1}}
\end{equation}  
\begin{equation}
trans([ \alpha	, \beta ]) = \vb{(A-id)b+(id-[A,B])b+A(id-B)A^{-1}a }
\end{equation} 
\end{lemma}

\begin{proof}
	$ [ \alpha, \beta ](\vb{x}) = ( \alpha \beta \alpha^{-1} \beta^{-1} ) (\vb{x})$. \\
	Dato che $ \alpha : \vb{x} \longmapsto \vb{Ax + a}$ , allora sicuramente  $\alpha^{-1}: \vb{y} \longmapsto \vb{A^{-1}(y-a)}$.  \\
	Allo stesso modo, dato che $ \beta : \vb{x} \longmapsto \vb{Bx + b}$ , so che $\beta^{-1}: \vb{y} \longmapsto \vb{B^{-1}(y-b)}$. 
\begin{equation*}
\begin{split}
[ \alpha, \beta ](\vb{x}) = (\alpha \beta \alpha^{-1} \beta^{-1} ) (\vb{x})
& = (\alpha \beta \alpha^{-1})( \vb{B^{-1}(x-b)}) = \\ 
& = (\alpha \beta \alpha^{-1})( \vb{B^{-1}x- B^{-1}b }) = \\ 
& = (\alpha \beta)(\vb{A^{-1}( B^{-1}x- B^{-1}b -a)})= \\
& =  (\alpha \beta)(\vb{A^{-1} B^{-1}x- A^{-1}B^{-1}b -A^{-1}a})  = \\
& = (\alpha)(\vb{B(A^{-1} B^{-1}x- A^{-1}B^{-1}b -A^{-1}a) +b}) = \\
& = (\alpha)(\vb{BA^{-1} B^{-1}x-BA^{-1}B^{-1}b -BA^{-1}a +b}) = \\
& = \vb{A(BA^{-1} B^{-1}x-BA^{-1}B^{-1}b -BA^{-1}a +b) + a} = \\
& = \vb{A(BA^{-1} B^{-1}x-BA^{-1}B^{-1}b -BA^{-1}a +b) + a} = \\
& = \vb{ABA^{-1} B^{-1}x-ABA^{-1}B^{-1}b -ABA^{-1}a +Ab + a }
\end{split}
\end{equation*}
Quindi $ rot([ \alpha , \beta ]) = \vb{ABA^{-1} B^{-1} }$ e
\begin{equation*}
\begin{split}
trans([ \alpha, \beta ]) & = \vb{-ABA^{-1}B^{-1}b -ABA^{-1}a +Ab + a }= \\
& = \vb{Ab + a -ABA^{-1}B^{-1}b -ABA^{-1}a }= \\
& = \vb{(A-id)b + b + a -ABA^{-1}B^{-1}b -ABA^{-1}a} = \\
& = \vb{(A-id)b + (id-[A,B])b + a - ABA^{-1}a} = \\
& = \vb{(A-id)b + (id-[A,B])b + (id- ABA^{-1})a }= \\
& = \vb{(A-id)b + (id-[A,B])b + A(id- B)A^{-1}a} 
\end{split}
\end{equation*}	
\end{proof}

\section{"Misura" della componente di rotazione}
\begin{definition}
Comunque preso  $  \vb{A} \in O(n) $  definisco
\[ m(\vb{A}) = \max \bigg\{ \frac{\norm{\vb{Ax-x}}}{\norm{\vb{x}}} \bigg|  \vb{x} \in \mathbb{R}^{n}-\vb{0} \bigg\} \]
\end{definition}
Questa funzione stima quanto una data matrice ortogonale si comporta in modo diverso dalla matrice identità; descrive infatti quanto al massimo un vettore unitario viene "spostato" dall'azione di tale matrice. 
\begin{lemma}
La precedente è una buona definizione e inoltre vale
\[ m(\vb{A}) = \max \bigg\{ \norm{\vb{Ax-x}} \bigg|  \vb{x} \in \mathbb{R}^{n} \land \norm{\vb{x}} = 1 \bigg\} \]
\end{lemma}

\begin{proof}
$ \forall \vb{x} \in \mathbb{R}^{n}-\vb{0}$  $ \exists \vb{y} \in \mathbb{R}^{n}-\vb{0} $ tale che $\; \; \vb{x} = \lambda \vb{y} , \lambda \in \mathbb{R} \land \norm{\vb{y}} = 1 $. 

\[ \frac{\norm{\vb{Ax-x}}}{\norm{\vb{x}}} = \frac{\norm{\vb{A}\lambda \vb{y}-\lambda \vb{y}}}{\norm{\lambda \vb{y}}} =   \frac{|\lambda| \norm{\vb{Ay-y}}}{|\lambda| \norm{\vb{ y}}} = \norm{\vb{Ay-y}} \]

L'insime $\vb{x} \in \mathbb{R}^{n} \land \norm{\vb{x}} = 1 $  è un compatto in  $\mathbb{R}^{n}$, quindi per il teorema di Weierstrass esiste una $\vb{x} \in \mathbb{R}^{n} : \norm{\vb{x}} = 1$ che mi verifica il max. 
\end{proof}

\begin{lemma}
Comunque preso $ \vb{x} \in \mathbb{R}^{n}$  vale  \[ \norm{\vb{Ax-x}} \leq m(\vb{A}) \norm{\vb{x}} \]
\end{lemma}

\begin{proof}
Infatti vale $\forall \vb{x} \in \mathbb{R}^{n}-\vb{0}$  e vale anche per  $\vb{x}=0$
\end{proof}
Una volta defiinta la funzione $m$, è possibile definire il sottoinsieme di $\mathbb{R}^n$ che contiene tutti e soli i vettori che realizzano il massimo nella sua definizione. 
\begin{definition}
\[ E_{\vb{A}} := \{ \vb{x} \in \mathbb{R}^{n} :  \norm{\vb{Ax - x}} = m(\vb{A}) \norm{\vb{x}}\} \]
\end{definition}

\begin{lemma}
$E_A$ è un sottospazio di $\mathbb{R}^{n}$ non banale ed A-invariante
\end{lemma} 
\begin{proof}
\begin{itemize}
\item $\vb{0} \in E_{\vb{A}} $  
\item $\forall \vb{x} \in E_{\vb{A}}, \forall \lambda \in \mathbb{R}$ \\
$\vb{x} \in E_{\vb{A}} \Longrightarrow \norm{\vb{Ax-x}} = m(\vb{A})\norm{\vb{x}} \Longrightarrow \norm{\vb{A} \lambda \vb{x} -\lambda \vb{x}}= | \lambda | m(\vb{A}) \norm{\vb{x}}  \Longrightarrow \lambda \vb{x} \in E_{\vb{A}} $
\item $\forall \vb{x, y} \in E_{\vb{A}} $ si vuole verificare che $\vb{x+y, x-y} \in E_{\vb{A}}$ \\
$\vb{x} \in E_{\vb{A}} \Longrightarrow \norm{\vb{Ax-x}} = m(\vb{A})\norm{\vb{x}}$ \\
$\vb{y} \in E_{\vb{A}} \Longrightarrow \norm{\vb{Ay-y}} = m(\vb{A})\norm{\vb{y}}$ 
\begin{equation} 
\begin{split}
& \norm{\vb{A(x+y) - (x+y)}}^2 + \norm{\vb{A(x-y) - (x-y)}}^2  = \\
& = \norm{\vb{Ax - x + Ay -y}}^2 + \norm{\vb{Ax - x -(Ay -y)}}^2 = \\
& = 2 \norm{\vb{Ax-x}}^2 + 2 \norm{\vb{Ay -y}}^2  = 2( \norm{\vb{Ax-x}}^2 + \norm{\vb{Ay-y}}^2) = \\
& = 2m(\vb{A})^2(\norm{\vb{x}}^2 + \norm{\vb{y}}^2) 
\end{split}
\end{equation}
Dalla precedente catena di uguaglianze segue
\begin{equation} 
\begin{split}
2m(\vb{A})^2(\norm{\vb{x}}^2 + \norm{\vb{y}}^2) & = \norm{\vb{A(x+y) - (x+y)}}^2 + \norm{\vb{A(x-y) - (x-y)}}^2 \leq \\
& \leq m(\vb{A})^2(\norm{\vb{x+y}}^2 + \norm{\vb{x-y}}^2) = \\ 
& = m(\vb{A})^2(\norm{\vb{x+y}}^2 + \norm{\vb{x-y}}^2)  = \\
& = 2m(\vb{A})^2(\norm{\vb{x}}^2 + \norm{\vb{y}}^2)
\end{split}
\end{equation}
Quindi il segno di disuguaglianza in (4) è un'uguaglianza, in particolare 
\[   \norm{\vb{A(x+y) - (x+y)}}^2 + \norm{\vb{A(x-y) - (x-y)}}^2  = m(\vb{A})^2(\norm{\vb{x+y}}^2 + \norm{\vb{x-y}}^2)  \]
Spostando i termini da parte a parte ottengo
\[ \norm{\vb{A(x+y) - (x+y)}}^2 - m(\vb{A})^2 \norm{\vb{x+y}}^2=  m(\vb{A})^2 \norm{\vb{x-y}}^2 - \norm{\vb{A(x-y) - (x-y)}}^2 \]
E' possibile poi scomporre le differenze di quadrati a sinistra e destra dell'uguaglianza ottenendo
\begin{equation} 
\begin{split}
\big( & \norm{\vb{A(x+y) - (x+y)}}  + m(\vb{A})\norm{\vb{x+y}} \big) \big( \norm{\vb{A(x+y) - (x+y)}} - m(\vb{A}) \norm{\vb{x+y}} \big) = \\
& -\big( \norm{\vb{A(x-y) - (x-y)}} + m(\vb{A})\norm{\vb{x-y}} \big) \big( \norm{\vb{A(x-y) - (x-y)}} - m(\vb{A}) \norm{\vb{x-y}} \big).
\end{split}
\end{equation}
Distinguiamo alcuni casi: 
\begin{itemize}
\item Se $\vb{A} = \vb{id} \Rightarrow m(\vb{A}) = 0 \Rightarrow E_{\vb{A}} = \mathbb{R}^{n}$  ed in quel caso la dimostrazione è conclusa.
\item Se $\vb{x} = \vb{y}$ o $\vb{x} = \vb{-y}$  la dimostrazione è conclusa per il punto precedente (rispettivamente con $\lambda = +1$ e $\lambda = -1$). 
\item Nei restanti casi si osserva che nell'equazione (5) il primo fattore a sinistra dell'uguaglianza è strettamente positivo, mentre il secondo fattore è $\leq 0 $. Allo stesso modo a destra dell'uguaglianza è presente un meno che modifica il segno del prodotto; il primo fattore è strettamente positivo ed il secondo è $\leq 0 $. \\ L'uguaglianza in (5) deve quindi per forza coincidere con $0=0$ e questo implica 
\begin{equation*}
  \left\{
    \begin{aligned}
      & \norm{\vb{A(x+y) - (x+y)}} = m(\vb{A})\norm{\vb{x+y}} \\
      & \norm{\vb{A(x-y) - (x-y)}} = m(\vb{A})\norm{\vb{x-y}} 
    \end{aligned}
  \right.
\end{equation*}

$\Longrightarrow \vb{x+y, x-y} \in E_{\vb{A}} $ 
\end{itemize}  
La dimostrazione del fatto che $E_{\vb{A}}$ è un sottospazio vettoriale di $\mathbb{R}^n $ è così conclusa.
\item Mostriamo che $E_{\vb{A}}$ è non banale.\\
L'insieme degli $\vb{x} \in \mathbb{R}^{n} \land \norm{\vb{x}} = 1$ è un compatto; l'applicazione $ \vb{x} \mapsto \norm{\vb{Ax-x}}$ è continua in quanto composizione di funzioni continue, quindi sicuramente esiste un \\ $\vb{x} \in \mathbb{R}^{n} \land \norm{\vb{x}} = 1$  che  verifica il massimo. \\
In particolare $\vb{x} \neq \vb{0}$  (perché ha norma 1) e 
\[ \norm{\vb{Ax - x}} = m(\vb{A}) = m(\vb{A})\norm{\vb{x}} \Longrightarrow \vb{x} \in E_{\vb{A}}\]
\item Mostriamo che $E_{\vb{A}}$ è $\vb{A}$-invariante, ovvero che  $\forall \vb{x} \in E_{\vb{A}} \Longrightarrow \vb{Ax} \in E_{\vb{A}}$ \\
$\vb{x} \in E_{\vb{A}} \Longrightarrow \norm{\vb{Ax-x}} = m(\vb{A})\norm{\vb{x}} $ \\
Dato che $\vb{A} \in O(n)$ vale la seguente catena di uguaglianze:\\
$ \norm{\vb{A(Ax) - Ax }} =  \norm{\vb{A(Ax-x)}} = \norm{\vb{Ax - x}} = m(\vb{A})\norm{\vb{x}} = m(\vb{A})\norm{\vb{Ax}} \Longrightarrow \vb{Ax} \in E_{\vb{A}}$
\end{itemize} 
\end{proof}
Il seguente lemma è un noto risultato relativo alle scomposizioni ortogonali di spazi vettoriali. 
\begin{lemma}
$E_{\vb{A}}$ sottospazio vettoriale di $\mathbb{R}^n $ non banale, è possibile quindi definire il suo complemento ortogonale $E^{\perp}_{\vb{A}} \neq \mathbb{R}^n$ ed anche questo è $\vb{A}$-invariante. 
\end{lemma} 

\begin{definition}

\[ m^{\perp}(\vb{A}) = \begin{cases} 
      \max \bigg\{ \frac{\norm{\vb{Ax-x}}}{\norm{\vb{x}}} \bigg|  \vb{x} \in E^{\perp}_{\vb{A}} - \vb{0} \bigg\}  &  se   E^{\perp}_{\vb{A}} \neq {\vb{0}} \\
      0 & se   E^{\perp}_{\vb{A}} = {\vb{0}} \\
      
   \end{cases}
\]

\end{definition}

\begin{lemma}

\begin{equation}
\begin{split}
m^{\perp}(\vb{A})  & < m(\vb{A}) \; \; \; \; \; \; \; \; \; se \; \vb{A \neq id}  \\
 m^{\perp}(\vb{A})  & = m(\vb{A}) = 0 \; \; \;  se \; \vb{A = id}
\end{split}
\end{equation}

\end{lemma}

\begin{proof}
\begin{itemize}
\item Se $\vb{A = id} \Longrightarrow m(\vb{A}) = 0 \land E_{\vb{A}} = \mathbb{R}^n \Longrightarrow E^{\perp}_{\vb{A}} = {0} \Longrightarrow m^{\perp}(\vb{A}) = 0$

\item Se $\vb{A \neq id}$ 
\[ m(\vb{A}) = \max \bigg\{ \frac{\norm{\vb{Ax-x}}}{\norm{\vb{x}}} \bigg|  \vb{x} \in \mathbb{R}^{n}-\vb{0} \bigg\} 
\geq \max \bigg\{ \frac{\norm{\vb{Ax-x}}}{\norm{\vb{x}}} \bigg|  \vb{x} \in E^{\perp}_{\vb{A}} - \vb{0} \bigg\} =  m^{\perp}(\vb{A}) \]
Se valesse $m(\vb{A}) = m^{\perp}(\vb{A}) \Longrightarrow \exists \vb{x} \in E^{\perp}_{\vb{A}} - \vb{0} : \norm{\vb{Ax-x}} = m(\vb{A})\norm{\vb{x}} \Longrightarrow \vb{x} \in E_{\vb{A}}$  ma questo implica $\vb{x} \in E_{\vb{A}} \cap E^{\perp}_{\vb{E}} = {\vb{0}} \Longrightarrow \vb{x = 0} $;  \\
questo è però assurdo perché abbiamo imposto che $x\vb{ \neq 0}$ 
\end{itemize}
\end{proof}

\begin{lemma}
$\forall \vb{x} \in \mathbb{R}^n $ posso scrivere $\vb{x}$ in modo unico in decomposizione ortogonale. 
\[ \vb{x} = \vb{x}^E + \vb{x}^{\perp} \; \mathit{dove} \; \vb{x}^E \in E_{\vb{A}} \; \mathit{e} \; \vb{x}^{\perp} \in E^{\perp}_{\vb{A}} \]
\'E inoltre immediato verificare le seguenti proprietà:
\[ \norm{\vb{Ax}^E - \vb{x}^E }= m(\vb{A})\norm{\vb{x}^E} \] \[\norm{\vb{Ax}^{\perp} - \vb{x}^{\perp}} \leq m(\vb{A}) \norm{\vb{x}^{\perp}} \]
\end{lemma}

Il seguente risultato ci permette di stimare la "misura" del commutatore di due matrici a partire dalla misura delle stesse. 
\begin{lemma}
 $\forall \vb{ A, B} \in O(n)$ vale la disuguaglianza 
 \[ m([\vb{A,B}]) \leq 2m(\vb{A})m(\vb{B}) \]
\end{lemma}

\begin{proof}
\begin{equation*}
\begin{split}
[\vb{A,B}] - \vb{id} & = \vb{ABA^{-1}B^{-1} - id } = \\
& = \vb{(AB -BA)A^{-1}B^{-1} } = \\
& = [ \vb{ (A-id)(B-id) - (B-id)(A-id)}] \vb{A^{-1}B^{-1}} \\
& = \vb{( A-id)(B-id) A^{-1}B^{-1} -  (B-id)(A-id) A^{-1}B^{-1}}
\end{split}
\end{equation*}  
Dalla precedente catena di uguaglianze segue, scelto comunque $\vb{x} \in \mathbb{R}^n $
 \begin{equation*} 
\begin{split}
\norm{\vb{([A,B] - id )x}} & \leq \norm{\vb{ ( A-id)(B-id) A^{-1}B^{-1}x}} +  \norm{\vb{(B-id)(A-id) A^{-1}B^{-1}x}} \leq \\
& \leq m(\vb{A}) \norm{\vb{(B-id) A^{-1}B^{-1}x}} + m(\vb{B}) \norm{\vb{(A-id) A^{-1}B^{-1}x}} \leq \\ 
& \leq m(\vb{A})m(\vb{B})\norm{\vb{A^{-1}B^{-1}x}} + m(\vb{B})m(\vb{A}) \norm{\vb{A^{-1}B^{-1}x}}
\end{split}
\end{equation*}
Dato che $ \vb{A,B} \in O(n)$ è possibile scrivere $\norm{\vb{A^{-1}B^{-1}x}} = \norm{\vb{x}} $ , quindi 
\begin{equation*}
  \norm{\vb{([A,B] - id )x}}  \leq 2m(\vb{A})m(\vb{B})\norm{\vb{x}}
\end{equation*} 
Quindi segue immediatamente la tesi. 
\end{proof}

\chapter{Primo teorema di Bieberbach}
In questo capitolo sezione viene dimostrato il primo teorema di Bieberbach. \\
Nella prima sezione si dà la definizione precisa di gruppo cristallografico e viene esplicitato il contenuto del teorema nella forma in cui verrà dimostrato. \\
Nella seconda sezione sono presentati e dimostrati alcuni sottoteoremi che concorrono alla dimostrazione del primo teorema di Bieberbach. \\ 
Infine nella sezione finale verranno connesse le varie proposizioni in una dimostrazione del teorema vero e proprio.  \\

\section{Enunciati}
In questo elaborato utilizzo la seguente definizione di gruppo cristallografico:
\begin{definition}
Sia $\Gamma$ un sottogrupo di $Isom(n)$, dico che è un gruppo cristallografico $n$-dimensionale se valgono le seguenti condizioni:
\begin{enumerate}
	\item $ \forall t \in \mathbb{R} : t > 0 $  esistono solo un numero finito di $\alpha = \vb{(A, a)} \in \Gamma $  tali che  $\norm{a} \leq t$
	\item $ \exists d \in \mathbb{R}: \forall \vb{x} \in \mathbb{R}^n   \exists \alpha = \vb{(A,a)} \in \Gamma : \norm{\vb{a-x}}\leq d $
\end{enumerate}
\end{definition}
La condizione 1. implica che il gruppo agisce in modo propriamente discontinuo sullo spazio topologico dato da $\mathbb{R}^n$ con la topologia euclidea, mentre la condizione 2. significa che l'azione di $\Gamma$ ha dominio fondamentale limitato. Ricoridamo che, dato uno spazio topologico $X$ ed un gruppo $G$ che agisce su di esso, un dominio fondamentale è un sottoinsieme di $X$ che contiene uno ed un solo punto di ogni orbita dell'azione.\\
\begin{theorem}
Ogni gruppo cristallografico $n-$dimensionale contiene $n$ translazioni linearmente indipendenti
\end{theorem}
\newpage 
\section{Mini Bieberbach e caratterizzazione delle translazioni} 
\begin{theorem}{Mini Bieberbach.} \\
Sia $\Gamma$ un gruppo cristallografico di $\mathbb{E}^n$. Comunque scelto $ \vb{u} \in \mathbb{R}^n : \norm{\vb{u}} = 1$ ,\\
$  \forall \epsilon ,\delta >0 $ $   \exists \alpha = \vb{(A,a)} \in \Gamma $ che soddisfa 
\begin{enumerate}
\item $\vb{a} \neq 0$ 
\item $ \angle (\vb{u, a}) \leq \delta $
\item $ m(\vb{A}) \leq \epsilon$ 
\end{enumerate}
\end{theorem}

\begin{proof}
Per la seconda proprietà dei gruppi cristallografici sappiamo che \\
$ \exists d \in \mathbb{R} : \forall k \in \mathbb{N} $ $  \exists \beta_k = \vb{( B_k, b_k)} \in \Gamma  $ tale che 
\[  \norm{\vb{b_k - ku }} \leq d \]
Se $k \longrightarrow \infty$ allora sicuramente $ \norm{\vb{b_k}} \longrightarrow \infty$.  \\
Infatti se, per assurdo, questo non fosse vero $\exists M \in \mathbb{R} : \norm{\vb{b_k}} \leq M $   $  \forall k \in \mathbb{N} $
\[ d \geq \norm{\vb{b_k - ku}} \geq | \norm{\vb{b_k}} - k | \]
Nella disequazione  precedente, se la successione è limitata allora l'ultimo termine diverge ma questo è assurdo perché è maggiorato da $d$. \\
Consideriamo ora la successione degli angoli fra i vettori $\vb{u}$ e $\vb{b}_k$, vogliamo mostrare che l'ampiezza tende a 0. 
\[ \angle (\vb{u , b_k}) = \angle (\vb{ku, b_k}) = : \theta_k \]
\begin{figure}[!h]
\centering
\label{fig:trangolo1}
\begin{tikzpicture}[line cap=round]
  \coordinate (O) at (0,0);
  \coordinate (A) at (6, 0);
  \coordinate (B) at (5,2.4);
  \coordinate (A-B) at ($(A)-(B)$);
  \draw[vector,myred] (O) -- (A) node[midway,below] {$k \vb{u}$};
  \draw[vector,myblue] (O) -- (B) node[midway,above left=-2] {$\vb{b_k}$};
  \draw[vector,mypurple] (A) -- (B) node[above right=-3] {$\vb{b_k}-k\vb{u}$};
  \pic[draw,"$\theta_k$",angle radius=20,angle eccentricity=1.25] {angle=A--O--B}; 
  \pic[draw,"$\gamma_k$",angle radius=20,angle eccentricity=1.25] {angle=B--A--O};  
\end{tikzpicture}
\centering
\caption{I vettori $k \vb{u}$, $\vb{b_k}$ e $\vb{b_k}-k\vb{u}$ formano un triangolo. \\ Chiamo $\gamma_k$ l'angolo contenuto fra $k \vb{u}$ e $\vb{b_k}-k\vb{u}$}
\end{figure} \\
Possiamo applicare il teorema dei seni al triangolo in figura ~\ref{fig:trangolo1}
\[ \frac{\norm{\vb{b_k}}}{sin(\gamma_k)} = \frac{\norm{\vb{b_k} - k\vb{u}}}{sin(\theta_k)}  \leq \frac{d}{sin(\theta_k)} \]
Si può ora riscrivere la disequazione in modo da ottenere
\[  sin(\theta_k) \leq \frac{sin(\gamma_k)d}{\norm{\vb{b_k}}} \longrightarrow 0 \Longrightarrow \theta_k \longrightarrow 0 \]

$O(n)$  è compatto e ${ \{\vb{B}_k \} }_{k \in \mathbb{N}}$ successione in  $O(n)$  ammette quindi almeno un punto di accumulazione per il teorema di Bolzano-Weierstrass. Si può quindi estarrre da ${ \{ \vb{B}_k \}}_{k \in \mathbb{N}}$  una sottosuccessione  convergente: $ { \{ \vb{A}_k \} }_{k \in \mathbb{N}} = { \{ \vb{B}_{k_j} \} }_{j \in \mathbb{N}}$ .  \\
\newpage
La funzione  $ m: O(n) \longrightarrow \mathbb{R}$  è continua in quanto composizione di funzioni continue ($ \vb{A} \mapsto \max{\norm{\vb{Ax-x}}} $), quindi con $i,j \in \mathbb{R} \longrightarrow \infty $ sicuramente $m(\vb{A}_j \vb{A}^{-1}_i) \longrightarrow m(\vb{id}) = 0 $ \\
Associata a questa sottosuccessione ho ovviamente una sottosuccessione di ${ \{\beta_k \}}_{k \in \mathbb{N}}$ che chiamo ${ \{\alpha_k \}}_{k \in \mathbb{N}} = \{ (\vb{A}_k , \vb{a}_k ) \}_{k \in \mathbb{N}}$. \\


Dato che valgono tutte le proprietà appena dimostrate, è immediato verificare che $\exists i,j \in \mathbb{N} $  tali che  $ i < j$  e che valgano contemporaneamente
\[ \begin{cases} \angle(u, \vb{a}_j ) \leq \frac{\delta}{2} \\
\norm{\vb{a}_i} \leq \frac{\delta}{4}\norm{\vb{a}_j} \\
m( \vb{A}_j \vb{A}^{-1}_i ) \leq \epsilon  \end{cases} \]
Considero l'isometria definita da $\alpha \in \Gamma$ come 
\[ \alpha = (\vb{A, a})  : x \longmapsto \alpha_j \alpha^{-1}_i \vb{x} = \vb{A}_j \vb{A}^{-1}_i \vb{x} + \vb{a}_j - \vb{A}_j \vb{A}^{-1}_i \vb{a}_i\]
Questa isometria verifica tutte le proprietà richieste dalla tesi

\begin{itemize}
\item $m(\vb{A}) = m(\vb{A}_j \vb{A}^{-1}_i) \leq \epsilon$ vero per precedente dimostrazione
\item Verifichiamo  $\vb{a}= \vb{a}_j - \vb{A}_j \vb{A}^{-1}_i \vb{a}_i \neq 0$ 
\[ \norm{\vb{a}_j - \vb{A}_j \vb{A}^{-1}_i \vb{a}_i} \geq \bigg| \norm{\vb{a}_j} - \norm{ \vb{A}_j \vb{A}^{-1}_i \vb{a}_i } \bigg| \geq \bigg| \norm{\vb{a}_j } + \norm{\vb{a}_i} \bigg| \geq \norm{\vb{a}_i} \bigg| \frac{4}{\delta} -1 \bigg| \]
$\norm{\vb{a}_i} \neq 0 \Longrightarrow \vb{a} \neq 0$
 
\item Verifichiamo $ \angle (\vb{u, a})  = \angle (\vb{u} , \vb{a_j - A_j A_i^{-1} a_i} ) \leq \delta $. Applicando la disuguaglianza triangolare sugli angoli si può scrivere
\begin{equation}
\angle (\vb{u} , \vb{a_j - A_j A_i^{-1} a_i} ) \leq \angle (\vb{u} , \vb{a_j}) + \angle(\vb{a_j}, \vb{a_j- A_j A_i^{-1} a_i} ) 
\end{equation}

Dato che $ \norm{\vb{a}_i} \leq \frac{\delta}{4}\norm{\vb{a}_j}$, la punta del vettore $\vb{a}_j - \vb{A_j A_i^{-1} a_i}$ cade all'interno di una palla n-dimensionale che ha come centro la punta di $\vb{a}_j$ e raggio $r = \frac{\delta}{4}\norm{\vb{a}_j}$\\
\begin{figure}[!h]
\centering
\label{fig:triangolo2}
\begin{tikzpicture}[line cap=round]
  \coordinate (O) at (0,0);
  \coordinate (A) at (9,0);
  \coordinate (B) at (1.5,1);
  \coordinate (C) at ($(O)-(B)$);
  \coordinate (U) at (1, -1.5);
  \coordinate (X) at ($(A)+(C)$);
  \draw[vector] (O) -- (A) node[midway,above] {$\vb{a}_j$};
  \draw[vector,red] (O) -- (B) node[above right = -2] {$\vb{A_j A_i^{-1} a_i}$};
  \draw (A) circle (2)node[right]{$r = \frac{\delta}{4}\norm{\vb{a}_j}$};
  \draw[vector, black] (O) -- (U) node[midway, below] {$ \vb{u} $};
  \draw[vector,red] (A) -- (X);
  \draw[vector] (O) -- (X)node[midway,below] {$\vb{a}_j - \vb{A_j A_i^{-1} a_i}$};
\end{tikzpicture}
\caption{AGGIUNGERE UNA CAPTION SENSATA}
\end{figure} 

\begin{minipage}{0.3\textwidth}
$\forall \rm P$ scelto all'interno della circonferenza \\ $ \widehat{PAO} \leq \widehat{TAO} $ 
\[ \frac{\overline{TO}}{sin( \widehat{TAO}) } = \frac{\overline{AO}}{sin( \widehat{ATO}) } \] 
Quindi
\[ sin(\widehat{PAO}) \leq sin( \widehat{TAO})= \frac{\overline{TO}}{\overline{AO}} = \frac{\delta}{4}\]
\end{minipage} \hfill
\begin{minipage}{0.6\textwidth}
\begin{tikzpicture}
  \def\r{2} % radius
  \def\q{-7} % distance center-external point q = |OQ|
  \def\x{{\r^2/\q}} % Q x coordinate
  \def\y{{\r*sqrt(1-(\r/\q)^2}} % Q y coordinate
  \coordinate (O) at (0,0); % circle center O
  \coordinate (Q) at (\q,0); % external point Q
  \coordinate (P) at (\x,\y); % point of tangency, P
  \coordinate (X) at (1, 1);
  \draw[blue,thick] (O) circle(\r);
  \draw[green,thick] (Q) -- (P);
  \draw[green,thick] (P) -- (O);
  \draw[green,thick] (O) -- (Q);
  \draw (O) -- (X);
  \draw (Q) -- (X);
  \fill(O) circle(0.05) node[below right] {O};
  \fill(Q) circle(0.05) node[below left] {A};
  \fill(P) circle(0.05) node[above=3] {T};
  \fill(X) circle(0.05) node[above=3,right=4] {P};
  \pic [draw, angle radius=4, angle eccentricity=4] {right angle = O--P--Q};
\end{tikzpicture}
\end{minipage}

So quindi che 
\begin{equation}
sin(\angle(\vb{a_j}, \vb{a_j- A_j A_i^{-1} a_i} ) ) \leq \frac{\delta}{4} \Longrightarrow \angle(\vb{a_j}, \vb{a_j- A_j A_i^{-1} a_i} ) \leq \frac{\delta}{4} + o \bigg( \frac{\delta^2}{16} \bigg) \leq \frac{\delta}{2}
\end{equation}
E (9) insieme con (8) e $ \angle(\vb{u}, \vb{a}_j) \leq \frac{\delta}{2}$ implica che 
\[ \angle (\vb{u, a}) \leq \delta \]
\end{itemize}


\end{proof}


\begin{theorem}
 Comunque scelta $\alpha \in \Gamma$:  $x \longmapsto \vb{Ax +a} $  tale per cui $m(\vb{A}) \leq \frac{1}{2} $, questa isometria è una traslazione pura 
\end{theorem}

\begin{proof}
 Se $m(\vb{A}) = 0 \Longrightarrow A = id \Longrightarrow \alpha$ è una traslazione pura. \\
 
 
\begin{minipage}{0.5\textwidth}
Fra le isometrie in $\Gamma$ che soddisfano la condizione  $ 0 < m(\vb{A}) \leq \frac{1}{2}$  scelgo quella che ha $\norm{\vb{a}}$ minimo (posso farlo perché vale la condizione (1) sugli elementi di un gruppo cristallografico). \\

So che $m(\vb{A}) > m^{\perp}(\vb{A})$ se  $\vb{A} \neq id$. \\
$\forall \vb{u} \in E_{\vb{A}}$ vettore unitario, \\ $\epsilon := \frac{1}{8} \bigg( m(\vb{A}) - m^{\perp}(\vb{A}) \bigg)$  \\ed applico il teorema Mini Bieberbach. \\
\[ \exists \beta \in \Gamma : m(\vb{B}) \leq \epsilon ; \vb{b} \neq 0 ; \angle(\vb{u, b})  \leq \delta \]
In particolare scelgo $\delta $ in modo da avere $\norm{\vb{b}^{\perp}} \leq \norm{\vb{b}^E}$, posso farlo come da figura. \\
Fra questi $\beta$ scelgo quello per cui $\norm{\vb{b}}$ è minimo ($\neq 0$, posso farlo per la prima proprietà dei gruppi cristallografici). \\
Osservo che, se $\beta$ non è una traslazione, allora \\
$ \norm{\vb{b}} \geq \norm{\vb{a}}$, questo perché $m(\vb{B}) \leq \frac{1}{8}m(\vb{A}) \leq \frac{1}{4}$ e $\alpha$ è stato scelto fra le isometrie in $\Gamma$ con in modo da minimizzare il modulo della componente traslatoria).  
\end{minipage} \hfill
\begin{minipage}{0.5\textwidth}
\begin{tikzpicture}
  \coordinate (O) at (0,0);
  \coordinate (A) at (3,5);
  \coordinate (B) at (0,4);
  \coordinate (C) at (1.5,5);
  \draw[dashed] (-3,0) -- (3,0)node[right] {$E^{\perp}_{\vb{A}}$};
  \draw[dashed] (1.5,0) |- (0,5);
  \draw[vector, red, thick] (O) -- (1.5,0)node[below] {$\vb{b}^{\perp}$};
  \draw[vector, red, thick] (O) -- (0,5)node[left] {$\vb{b}^{E}$};
  \draw[thick] (0,-2) -- (0,5.5)node[above] {$E_{\vb{A}}$};
  \draw[vector] (O) -- (B)node[right] {$\vb{u}$};
  \draw[vector] (O) -- (C)node[right] {$\vb{b}$};
  \draw[blue, thick] (0,0) -- (3,5);
  \draw[blue, thick] (0,0) -- (-3,5);
  \filldraw[black] (0,0) circle (2pt);
  \pic[draw,"$\delta$",angle radius=30,angle eccentricity=1.25] {angle=A--O--B};  
  \begin{scope}[on background layer]
		\fill[blue!20] (0,0) -- (-3,5) -- (3,5) -- cycle;
  \end{scope}
\end{tikzpicture}
\end{minipage}
Definisco una nuova isometria $ \tilde{\beta} := [\vb{\alpha,\beta}] \in \Gamma$  \\
So dal lemma 1.3 e dal lemma1.10 che
\[ m(\tilde{B} ) = m[\vb{A,B}] \leq 2m(\vb{A}) m(\vb{B}) \leq 2 \frac{1}{2}m(\vb{B}) = m(\vb{B}) \]
\[ trans( \tilde{\beta}) = \vb{(A-id)b + (id-[A,B])b + A(id-B)A^{-1}a} = \vb{(A-id)b + r}\]
Con $\vb{r} = \vb{(id-\tilde{B})b + A(id-B)A^{-1}a}$ e $\tilde{\vb{b}}^{\perp} =  \vb{(A-id)b^{\perp} + r}$.
\begin{itemize}
\item Se $\beta$  è una translazione $\Longrightarrow \vb{B = id = \tilde{B}} \Longrightarrow \vb{r} = 0$. 
\item Se $\beta$ non è una translazione $\Longrightarrow \norm{\vb{b}} \geq \norm{\vb{a}}$  e quindi \\
\[ \norm{\vb{r}} \leq m(\tilde{\vb{B}}) \norm{\vb{b}} + m(\vb{B}) \norm{\vb{a}} \leq \big( m(\tilde{\vb{B}}) + m(\vb{B}) \big) \norm{\vb{b}} \leq 2m(\vb{B})(\vb{b^{E}+ b^{\perp}}) < 4 m(\vb{B}) \norm{\vb{b}^{E}} \leq \frac{1}{2} \big( m(\vb{A}) - m^{\perp}(\vb{A}) \big)\norm{\vb{b}^{E}} \]
\end{itemize}
In entrambi i casi ho che $\norm{\vb{r}} < \frac{1}{2} \big( m(\vb{A}) - m^{\perp}(\vb{A}) \big)\norm{\vb{b}^{E}}$. \\
Posso scrivere 
\[ \tilde{\vb{b}}^{E} -(\vb{A-id) b}^{E} - \vb{r}^{E} = (\vb{A-id) b}^{\perp} + \vb{r}^{\perp} - \tilde{\vb{b}}^{\perp} = 0 \]
Usando la caratterizzazione $\norm{\vb{b}^{\perp}} \leq \norm{\vb{b}^E}$  ottengo
\[ \norm{\tilde{\vb{b}}^{\perp}} \leq  m^{\perp}(\vb{A}) \norm{\vb{b}^{\perp}} + \norm{\vb{r}^{\perp}} < m^{\perp}(\vb{A}) \norm{\vb{b}^{E}} + \frac{1}{2} \big( m(\vb{A}) - m(\vb{A}^{\perp}) \big)\norm{\vb{b}^{E}}  \] 
Sommando a destra ottengo
\[ \norm{\tilde{\vb{b}}^{\perp}} < \frac{1}{2} \big( m(\vb{A}) + m(\vb{A}^{\perp}) \big) \]
D'altro canto, utilizzanzo la disuguaglianza triangolare inversa, posso scrivere
\[ \norm{\vb{b}^{E}} = \norm{(\vb{A-id) b}^{E} + \vb{r}^{E}} \geq \bigg| m(A) \norm{\vb{b}^E} - \norm{\vb{r}}\bigg| > m(\vb{A}) \norm{\vb{b}^E} - \frac{1}{2} \bigg( m(\vb{A}) - m^{\perp}(\vb{A})\bigg)\norm{\vb{b}^E} =  \frac{1}{2} \big( m(\vb{A}) + m(\vb{A}^{\perp}) \big)\]
In particolare quindi 
\[ \norm{\vb{b}^{E}} > \norm{\vb{b}^{\perp}}\]
So inoltre che 
\[ \norm{\tilde{\vb{b}}} \leq m(\vb{A}) \norm{\vb{b}} + \norm{\vb{r}} < m(\vb{A})\norm{\vb{b}} + \frac{1}{2} \big( m(\vb{A}) - m^{\perp}(\vb{A}) \big)\norm{\vb{b}^{E}} \leq  \norm{\vb{b}} \bigg( \frac{1}{2} + \frac{1}{4}\bigg) < \norm{\vb{b}} \]
Ho un assurdo perché $\beta$ era stato scelto fra tutti quelli che soddisfavano le condizioni in modo da minimizzare la norma di $b$.


\end{proof}

\section{Dimostrazione del primo teorema di Bieberbach}

Posso quindi concludere la dimostrazione del primo teorema di Bieberbach.

\begin{proof}
Considero la base standard di $\mathbb{R}^N$; se applico ad ognuno dei vettori di tale base il teorema Mini Bieberbach con $\epsilon = \frac{1}{2}$ e $\delta = 1/N \forall N \in \mathbb{N}$, ottengo $n$ successioni di elementi in $\Gamma$ che sono translazioni pure (per il teorema 6.2). 
Se passo al limite per $N \longrightarrow \infty$ so che ognuna di queste successioni converge ad un elemento in $\Gamma$ \\
 (NON è VEROOOOOOOO!!! come faccio qui? devo far vedere che converge ad una direzione, ma non saprei come farlo. Inoltre come faccio a dire che se tot vettori distano ognuno meno di $\delta $ da un vettore della base standard, allora questi sono linearmente indipendenti? posso fare il limite per n --> infinito? non convergono necessariamente i moduli)
\end{proof}

\chapter{Secondo teorema di Bieberbach}

\begin{definition}
Un reticolo $L$ è un gruppo cristallografico che contiene solo translazioni. \\
Gli elementi di $L$ possono essere identificati con i vettori di $\mathbb{R}^n$ corrispondenti alla propria componenete di translazione e vengono chiamati punti di reticolo. 
\end{definition}

Dato un elemento $\omega \in L$, per abuso di notazione scriveremo $\omega = trans(\omega) = \vb{w}$. 
Enuncio il seguente risultato senza disubmostrarlo.
\begin{lemma}
Ogni reticolo $L$ di dimensione $n$ è isomorfo a $\mathbb{Z}^n$. \\
Di conseguenza $L$ è abeliano e la distanza minima fra due punti di reticolo coincide con la lunghezza del minimo vettore non nullo in $L$.  
\end{lemma}

\begin{lemma}
Sia $L$ un reticolo in $\mathbb{E}^n$ i cui vettori abbiano distanza a coppie $\geq 1$. \\
Sia $\rho >0$, chiamo $P(\rho)$ il numero di punti di reticolo in $L$ con distanza dall'origine $\leq \rho$. 
\[ P(\rho) \leq (2 \rho +1)^n\]
\end{lemma}
\begin{proof}
Per ogni punto di reticolo a distanza inferiore o uguale di $\rho$ dall'origine posso considerare una palla aperta $n$-dimensionale centrata in esso e di raggio $\frac{1}{2}$. Queste palle sono sicuramente disgiunte in quanto la distanzza fra due punti del reticolo è superiore al doppio dei raggi; sono inoltre tutte contenute nella palla $n$-dimensionale centrata nell'orgine di raggio $ \rho + \frac{1}{2}$. 
Il volume della palla centrata nell'origine è sicuramente superiore alla somma dei volumi delle singole palle di raggio $\frac{1}{2}$, confrontando i volumi ottengo

\[ P(\rho) \bigg(  \frac{1}{2} \bigg) \leq \bigg( \rho + \frac{1}{2} \bigg) \]
\[ P(\rho) \leq \bigg( 2 \rho + 1 \bigg)\]
\end{proof}

\begin{lemma}
Sia $L$ un reticolo in $\mathbb{E}^n$ i cui vettori abbiano distanza a coppie $\geq 1$. \\ Consideriamo un sottospazio lineare di $\mathbb{R}^n$ generato da $k$ vettori $\vb{w}_i \in L$ con $i = 1,... ,k$. \\
Se un punto di reticolo $\vb{w} \in L$ non è contenuto in $E$, allora la sua componenete in $E^{\perp}$ è tale che 
\[ \norm{\vb{w}^{\perp}} \geq \bigg( 3 +  \sum_{i = 1}^{k} \norm{\vb{w}_i} \bigg)^{-n} \]
\end{lemma}

\begin{proof}
Sia \hfill \[ N = \left\lfloor
\bigg( 3 +  \sum_{i = 1}^{k} \norm{\vb{w}_i} \bigg)^{n}
\right\rfloor\] \\
Suppongo per assurdo che $0 < \norm{\vb{w}^{\perp}}  < \frac{1}{N}$. \\
In questa situazione i vettori $j \vb{w}$ con $j = 0, ..., N$ hanno distanze da $E$ inferiori a 1. \\
Aggiungendo ad ognuno di questi una combinazione lineare di ${\{\vb{w}_i\}}_{1 \leq i \leq k }$ posso modificarne la componenete in $E$ senza andare a toccare la componente perpendicolare. \\
In particolare scelgo la combinazione in modo che $j \vb{w}^{E} \leq \frac{1}{2} \bigg( \sum_{i = 1}^{k} \norm{\vb{w}_i} \bigg)$     $\forall j = 0, ..., N$. 
Posso farlo in quanto (???????????????????????????????????????? devo rimanere nel reticolo, quindi dovrei fare combinazioni lineari intere, se fossero combinazioni lineari generiche sarebbe immediato; come lo faccio vedere?). \\
Questi $N+1$ punti di reticolo hanno quindi una distanza dall'origine inferiore a alla somma delle loro distanze da $E$ ed $E^{\perp}$, in quanto l'origine appartiene ad entrambi i sottospazi.
\[ d(j\vb{w}, \underline{0} ) \leq 1 + \frac{1}{2} \bigg( \sum_{i = 1}^{k} \norm{\vb{w}_i} \bigg) \]
Questa è una contraddizione al lemma (8.2) in quanto ho 
\[ N+1 \leq P \Bigg( 1 + \frac{1}{2} \bigg( \sum_{i = 1}^{k} \norm{\vb{w}_i} \bigg) \Bigg)  \leq \Bigg( 3 +  \sum_{i = 1}^{k} \norm{\vb{w}_i} \Bigg)^{-n} \leq  \frac{1}{N} \]
E non esistono $N \in \mathbb{N}^{+}$ che mi soddisfino questa disequazione.
\end{proof}


\begin{definition}
Sia $\Gamma$ un gruppo cristallografico $n$-dimensionale, definisco $L(\Gamma)<G$ come il sottogruppo che contiene tutte e sole le translazioni in $\Gamma$.
\end{definition}

La precedente è una buona definizione in quanto la composizione di due translazioni è una translazione. 

\begin{lemma}
Comunque scelte $\alpha \in \Gamma$ e $\vb{w} \in L(\Gamma) \Longrightarrow \vb{Aw} \in L(\Gamma)$ 
\end{lemma}

\begin{proof}
$w = trans(\omega)$ con $\omega \in L(\Gamma)$.
Considero $\alpha \omega \alpha^{-1} \in \Gamma$; questa è una translazione di vettore $\vb{Aw}$. Quindi $\vb{Aw} \in L(\Gamma)$
\end{proof}

\begin{definition}
Un gruppo cristallografico $\Gamma$ è detto normale se:
\begin{enumerate}
\item i vettori che sono contenuti in $L(\Gamma)$ hanno distanze a coppie $\geq 1$
\item $L(\Gamma)$ contiene $n$ vettori unitari linearmente indipendenti
\end{enumerate}
\end{definition}

\begin{theorem}
Ogni gruppo cristallografico $\Gamma$ è isomorfo ad un gruppo cristallografico normale
\end{theorem}

\begin{proof}
Per la proprietà (1) dei gruppi cristallografici so che esiste un vettore in $L(\Gamma)$ di lunghezza minima. Scalo per omotetia tutti i vettori in $\Gamma$ in modo da avere tale vettore di norma $1$. \\
Suppongo per induzione che $L(\Gamma)$ soddisfi la condizione (1) e che $\exists \vb{w}_i \in L(\Gamma)$ con $1 \leq i \leq k < n $ linearmente indipendenti tali che $\norm{\vb{w}_i} = 1 $.  
Chiamo $E$ il sottospazio vettoriale di dimensione $k$ di $\mathbb{R}^n$ generato da essi.\\
Voglio dimostrare che esiste un gruppo $\Gamma ' \cong \Gamma$ tale che $L(\Gamma')$ soddisfi la condizione (1) e che contenga $k+1$ vettori linearmente indipendenti. \\
\begin{itemize}
\item Se $\exists \alpha \in \Gamma $ e $1 \leq i \leq k: \vb{Aw}_i \notin E$ allora, grazie al lemma 2.0.4 so che $\vb{Aw}_i \in L(\Gamma)$, inoltre $\norm{\vb{Aw}_i} = 1$ dato che $A \in O(n)$, quindi $\vb{Aw}_i$ è già il $k+1$-esimo vettore cercato.
\item Se, al contrario, tutte le componenti di rotazione delle isometrie in $\Gamma$ lasciano $E$ invariato, allora queste lasciano invariato anche $E^{\perp}$. \\
Definisco la trasformazione affine 
\[ \Phi_{\nu} (x^E + x^{\perp} )=  x^E + \nu x^{\perp} , \; \nu > 0  \; \forall x \in \mathbb{R}^n\]
Scelto comunque $\alpha$ con componente rotatoria $\vb{A}$, dato che le componenti rotatorie di $\Gamma$ lasciano invariate $E$ ed $E^{\perp}$ vale
\[ ( \Phi_{\nu}  \circ \vb{A})(\vb{x}) = \Phi_{\nu} (\vb{Ax})  = \Phi_{\nu} (\vb{Ax}^E +\vb{Ax }^{\perp})  = \vb{Ax}^E +\nu \vb{Ax }^{\perp} = \vb{Ax}^E +\vb{A(\nu x) }^{\perp} = \vb{A} ( \vb{x}^E +\nu \vb{x }^{\perp})  = ( \vb{A} \circ \Phi_{\nu} ) (x)\]
Quindi la funzione affine $\Phi_{\nu}$ commuta con la parte rotazionale delle isometrie in $\Gamma$. \\
Definisco il gruppo $\Gamma_{\nu} := \Phi_{\nu} \Gamma \Phi_{\nu}^{-1} < Aff(\mathbb{R}^n)$. Questo è coniugato a $\Gamma$ nel gruppo affine (ed è quindi isomorfo a $\Gamma$), ma è effettivamente anche un gruppo di isometrie in quanto 
\[ \forall \alpha_{\nu} \in \Gamma_{\nu} \; \exists \alpha \in \Gamma : \; \; \alpha_{\nu} =  \Phi_{\nu} \alpha \Phi_{\nu}^{-1}  \]
\[ rot( \alpha_{\nu}) = rot(\Phi_{\nu} \alpha \Phi_{\nu}^{-1}) = rot(\alpha) \in O(n)\]

Se $\beta_{\nu} \in L(\Gamma_{\nu}) $ con  $ \beta_{\nu} = \Phi_{\nu} \beta \Phi_{\nu}^{-1} \Longrightarrow rot(\beta_{\nu}) = rot(\Phi_{\nu} \beta \Phi_{\nu}^{-1}) = rot(\beta) = \vb{id}  \Longrightarrow \beta \in L(\Gamma)$. \\ Scrivo $\beta \vb{x} = \vb{x} + \vb{b}  $. 
\[\beta_{\nu} \vb{x} = (\Phi_{\nu} \beta \Phi_{\nu}^{-1})(\vb{x}) = (\Phi_{\nu} \beta \Phi_{\nu}^{-1})(\vb{x}^E + \vb{x}^{\perp}) =  
 (\Phi_{\nu} \beta)(\vb{x}^E + \frac{1}{\nu}\vb{x}^{\perp}) =  \Phi_{\nu}(\vb{x}^E + \frac{1}{\nu}\vb{x}^{\perp} + \vb{b}) = \vb{x}^E + \vb{x}^{\perp} + \Phi_{\nu}(\vb{b}) \] 
Quindi  $L(\Gamma_{\nu})  = L(\Phi_{\nu} \Gamma \Phi_{\nu}^{-1}) = \Phi_{\nu}(L(\Gamma))$. \\
A questo punto se io considero $\nu >0 $ e suppongo che i vettori in  $L(\Gamma_{\nu})$ abbiano distanze a coppie $\geq 1$, posso applicare il teorema 2.0.3 utilizzando il sottospazio $E$. So quindi che $\forall \vb{w}_{\nu} = \in L(\Gamma_{\nu})$ tale che $\vb{w}_{\nu}^{\perp} \neq \vb{0}$ vale 
\[ \norm{\vb{w}_{\nu}^{\perp}} \geq (3 + k)^{-n}\]
So inoltre che $\exists \vb{w} \in \mathbb{R}^n : \vb{w}_{\nu} = \Phi_{\nu}(\vb{w})$, quindi posso scrivere
\[ \norm{\vb{w}_{\nu}^{\perp}} = \norm{\Phi_{\nu}(\vb{w})^{\perp}} = \norm{ (\vb{w}^{E} + \nu \vb{w}^{\perp})^{\perp}} = \norm{\nu \vb{w}^{\perp}} = \nu \norm{\vb{w}^{\perp}}\geq \frac{1}{(3 + k)^{n}} \]
Se scelgo $\nu$ molto piccolo, la disuguaglianza precedente non vale più, e di conseguenza non è vero che i vettori in  $L(\Gamma_{\nu})$ hanno distanze a coppie $\geq 1$. Esiste quindi un $\nu' > 0 $ minimo tale che la condizione (1) della definizione di gruppo cristallografico normale vale in $\Gamma_{\nu '}$.  \\
Dato che la trasformazione affine $\Phi_{\nu '}$ mantiene fissi tutti gli elementi $E$, il vettore più corto che si trova in $L(\Gamma_{\nu'}) - E  = \Phi_{\nu'}(L(\Gamma)) - E = \Phi_{\nu'}(L(\Gamma) - E)$ ha norma $1$ (PERCHé????????????????) \\
\end{itemize}
Quindi ho trovato  $\Gamma_{\nu'} \cong \Gamma$ che contiene $k+1$ vettori unitari linearmente indipendenti ed ho mostrato che $\Gamma_{\nu'}$ è un gruppo cristallografico normale.  \\
La dimostrazione si conclude per induzione. 
\end{proof}

\begin{theorem}
Ogni gruppo cristallografico normale è unicamente caratterizzato da una tabella di gruppo 
\end{theorem}

\begin{proof}
Sia $\Gamma$ il grupo cristallografico normale, siano fissati $n$ vettori $\vb{w}_1, ... \vb{w}_n \in L(\Gamma) $ tali che siano linearmente indipendenti e $\norm{\vb{w}_i} = 1  \; \; \forall i = 1, ..., n$.  \\
Considero il sottoreticolo  di $L(\Gamma)$
\[ L = \bigg\{ \sum_{i = 1}^{n} m_i \vb{w}_i \bigg| m_i \in \mathbb{Z}, i = 1, ..., n \bigg\} \]
$L$ è ovviamente un sottogruppo di $\Gamma$, faccio il quoziente $\quot{\Gamma}{L} $; da ognuno degli elementi di questo quoziente scelgo un rappresentante $\omega$ la cui componente translazionale $\vb{w} $ sia tale che 
\[ \norm{\vb{w}} \leq \frac{1}{2} \bigg( \sum_{i = 1}^{n} \norm{ \vb{w}_i)} \bigg) = \frac{n}{2}\]  
Per la prima proprietà dei gruppi cristallografici esistono solo un numero finito di questi rappresentanti, li chiamo $\omega_{n+1}, ... , \omega_N$. \\
Adesso $\forall \alpha \in \Gamma$, questo può essere scritto come 
\[ \alpha = \bigg( \sum_{i = 1}^{n} m_i \vb{w}_i \bigg) \omega_{\nu} \; \; \; \; n+1 \leq \nu \leq N\]
Se considero $\omega_i \in \Gamma$ con $i=1, ...,n$ come le translazioni di vettori $w_i \; \; i=1, ...,n$, posso caratterizzare unicamente la struttura di gruppo di $\Gamma$ a meno di isomorfismo come 
\begin{equation}
	 \omega_j \omega_k = \bigg( \sum_{i = 1}^{n} m_{ijk} \vb{w}_i\bigg) \omega_{\nu (j,k)}      \; \; \; \forall j,k =1, ...,N 
\end{equation}
$\Gamma$  è quindi unicamente determinato (a meno di isomorfismo) da $N, \omega_{\nu (j,k)}$ e $m_{ijk} \in \mathbb{Z}$.

\end{proof}

\begin{theorem}
I valori assoluti di $N, \omega_{\nu (j,k)}$ e $m_{ijk}$ sono maggiorati da un intero che dipende solo dalla dimensione $n$ dello spazio euclideo.
\end{theorem}
	

\end{document}
